%!TEX program = xelatex
\documentclass[11pt,a4paper]{article}

\usepackage{unicode-math}
\defaultfontfeatures{Ligatures={Common,TeX}}

\setmainfont{Libertinus Serif}
\setsansfont{Libertinus Sans}
\setmathfont{Libertinus Math}


\usepackage{metalogo}
\usepackage{booktabs}

\author{Matteo Sostero}

\title{\XeLaTeX{} Math Font Support}

\begin{document}

\newcommand{\U}[1]{\symbol{"#1}}

\maketitle

\centering

\begin{tabular}{ll}
\toprule
roman               & abcdefghijklmnopqrstuvwxyz                                                            \\
italic              & \emph{abcdefghijklmnopqrstuvwxyz}                                                     \\
capitals            & \uppercase{abcdefghijklmnopqrstuvwxyz}                                                \\
small capitals      & \textsc{abcdefghijklmnopqrstuvwxyz}                                                   \\
numbers             & {\addfontfeatures{Numbers={OldStyle,Proportional}}1234567890}                         \\
common ligatures    & ff fi fj fl ffi ffl ffj Th Qu                                                         \\
historic ligatures  & \addfontfeature{Ligatures={Rare,Historic}} Questo è strano assai! aesthetic           \\
quotations          & ‘bla’ “bla” «bla»                                                                     \\
diacritics          & è È é É ê Ê ë Ë ĕ Ĕ ē Ē ė Ė ȅ Ȅ ç ñ å                                                 \\
common symbols      & \& @ £ \$ € \% ‰ / \textbackslash{} [ ] \{ \} ( ) = ? ' \# ° \_* \~{} \^{} - -- --- · \\
rare symbols        & ± ½ ℝ ∂ ≈ ≠ ∞ § ♠ ♣ ♥ ♦ ‰ ¶ № ℅ ‼‽ ℅                                                  \\
kerning \& variants & thief? thief)                                                                         \\
special glyphs      & \U{E000} \U{E009} \U{E00A} \U{E040} \U{E001} \U{E002} \U{E003} \U{E13B} \U{E13C}      \\
\bottomrule
\end{tabular}

\medskip

\begin{itemize}
  \item This is some math text entered with math in the source:
  \[
    ∀X \left[ ∅ ∉ X ⇒ ∃f:X ⟶  ⋃ X\ ∀A ∈ X \left(f(A) ∈ A \right) \right]
  \]

  \item This is some math text entered with regular markup:
  \[
    \forall X \left[\emptyset \not\in X \Rightarrow \exists f:X \rightarrow
    \bigcup_{i=1}^n X\ \forall A \in X \left(f(A) \in A \right) \right  ]
  \]
\end{itemize}

\end{document}
