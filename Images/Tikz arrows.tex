\documentclass{beamer} %
\usetheme{CambridgeUS}
\usepackage[latin1]{inputenc}
\usepackage{tikz}
\usepackage{amsmath}
\usepackage{verbatim}
\usetikzlibrary{arrows,shapes}

\author{Author}
\title{Presentation title}

\begin{document}

% For every picture that defines or uses external nodes, you'll have to
% apply the 'remember picture' style. To avoid some typing, we'll apply
% the style to all pictures.
\tikzstyle{every picture}+=[remember picture]

% By default all math in TikZ nodes are set in inline mode. Change this to
% displaystyle so that we don't get small fractions.
\everymath{\displaystyle}

\begin{frame}
\frametitle{Structure of Demand}

\tikzstyle{na} = [baseline=-.5ex]

\begin{itemize}[<+-| alert@+>]
    \item Demand \tikz[na] \node[coordinate] (n1) {};
\end{itemize}

% Below we mix an ordinary equation with TikZ nodes. Note that we have to
% adjust the baseline of the nodes to get proper alignment with the rest of
% the equation.
\begin{equation*}
        \tikz[baseline]{\node[anchor=base] (t1){Z};} =
        \tikz[baseline]{\node[anchor=base] (t2){C};} +
        \tikz[baseline]{\node[anchor=base] (t3){I};} +
        \tikz[baseline]{\node[anchor=base] (t4){G};} +
        \tikz[baseline]{\node[anchor=base] (t5){X};} -
        \tikz[baseline]{\node[anchor=base] (t6){Im};} 
\end{equation*}

\begin{itemize}[<+-| alert@+>]
    \item Consumer spending 	\tikz[na]\node [coordinate] (n2) {};
    \item Investment 			\tikz[na]\node [coordinate] (n3) {};
    \item Government spending 	\tikz[na]\node [coordinate] (n4) {};
    \item Exports			 	\tikz[na]\node [coordinate] (n5) {};
    \item Imports			 	\tikz[na]\node [coordinate] (n6) {};
\end{itemize}

% Now it's time to draw some edges between the global nodes. Note that we
% have to apply the 'overlay' style.
\begin{tikzpicture}[overlay]
        \path[->]<1-> (n1) edge [bend left] (t1);
        \path[->]<2-> (n2) edge [bend right] (t2);
        \path[->]<3-> (n3) edge [bend right] (t3);
        \path[->]<4-> (n4) edge [bend right] (t4);
        \path[->]<5-> (n5) edge [bend right] (t5);
        \path[->]<6-> (n6) edge [bend right] (t6);
\end{tikzpicture}
\end{frame}
\end{document}