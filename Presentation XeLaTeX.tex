%!TEX program = xelatex
\PassOptionsToPackage{dvipsnames,x11names,svgnames}{xcolor}
\documentclass{beamer}                % Full presentation
% \documentclass[trans]{beamer}       % Transparency mode (no overlays)
% \PassOptionsToPackage{gray}{xcolor} % Force grey colours for printing
% \documentclass[handout]{beamer}     % Handout mode

\usepackage[xelatex]{mbeamer}

\usepackage{polyglossia}
\setdefaultlanguage{english}
\setotherlanguage{arabic}
\newfontfamily\arabicfont[Script=Arabic]{Times New Roman}

\usepackage{xeCJK} 
\setCJKmainfont{標楷體} % for brush stroke on Windows 
\setCJKfamilyfont{korm}{Gungsuh}

\setmonofont[Ligatures={NoCommon},Scale=MatchLowercase,UprightFont={* Medium}]{Source Code Pro}

\author{Matteo Sostero}
\authormail{Matteo Sostero}{m.sostero@sssup.it}
\institute{Sant'Anna School of Advanced Studies, Pisa}

\title{Title}
\subtitle{Subtitle}


\begin{document}


\begin{frame}
\includegraphics[scale=0.8]{./Images/SA_economics_logo_eng.pdf}
\maketitle
\end{frame}


\begin{frame}
\frametitle{Outline}
\tableofcontents
\end{frame}



\section{Blocks and lists}

\begin{frame}
\frametitle{Some lists and warnings}
\begin{itemize}[<+->]
\item First item
\item Second item
\item \alert{Important Warning!}
\item Hyperlink \highl{\hyperlink{Hayek-quote}{Hayek (1945)}} \only<4>{\hypertarget{Hayek-back}{}}
\end{itemize}
\pause[\thebeamerpauses]
\begin{alertblock}{Even more important warning!}
This is an important warning.
\end{alertblock}
\end{frame}


\begin{frame}
\frametitle{Why \LaTeX?}
\framesubtitle{From \url{http://www.ctan.org/what_is_tex.html}}
\begin{columns}[T]
\begin{column}{.46\textwidth}
\begin{block}{Output Quality}
\begin{itemize}
\item It has the best output.
\item It knows typesetting.
\end{itemize}
\end{block}
\uncover<3->{
  \begin{block}{Superior Engineering}
  \begin{itemize}
  \item It's fast.
  \item It's stable.
  \item It's not rigid (extensible).
  \item Plain text input.
  \item Many output types.
  \end{itemize}
  \end{block}
}
\end{column}

\begin{column}{.46\textwidth}
\uncover<2->{
  \begin{block}{Freedom}
  \begin{itemize}
  \item It's free.
  \item It runs anywhere.
  \end{itemize}
  \end{block}
}
\uncover<4->{
  \begin{block}{Popularity}
  \begin{itemize}
  \item It's the standard (in academia and science).
  \end{itemize}
  \end{block}
}
\end{column}
\end{columns}
\end{frame}



\section{Typesetting with \texorpdfstring{\XeLaTeX}{XeLaTeX}}

\begin{frame}
\frametitle{\XeLaTeX \ features}
  \begin{exampleblock}{Unicode support}
  \begin{columns}[T]
  \begin{column}{.45\textwidth}
  \renewcommand{\baselinestretch}{0.9}
  \begin{itemize}
  \item Arabic: \textarabic{العَرَبِية‎}
  \item Armenian: {\fontspec{Times New Roman} հայերէն }
  \item Cyrillic: ру́сский
  \item Greek: ελληνικά
  \item Hebrew: עִבְרִית
  \end{itemize}
  \end{column}
  \begin{column}{.55\textwidth}
  \renewcommand{\baselinestretch}{0.9}
  \begin{itemize}
  \item Traditional Chinese: 繁體字
  \item Simplified Chinese: 简化字 
  \item Hiragana: {\CJKnospace ひらがな}
  \item Katakana: カタカナ
  \item Hangul: {\CJKfamily{korm} \CJKspace 한글}
  \end{itemize}
  \end{column}
  \end{columns}
\end{exampleblock}
\begin{exampleblock}{Advanced font features}
\renewcommand{\baselinestretch}{0.9}
\begin{itemize}
  \item Real small capitals: \textsc{AaBbWwXxYyZz}
  \item Old style numerals: \addfontfeatures{Numbers={OldStyle,Proportional}} 0123456789
  \item Common ligatures: ff fi fj fl ffi ffl ffj Th Qu
  \item Historical ligatures: \addfontfeatures{Ligatures={Rare,Historic}} Quest \& action. Just historic.
  \item Font-specific glyphs: \symbol{"E000} \symbol{"E009} \symbol{"E00A} \symbol{"E040} \symbol{"E001}  \symbol{"E002}  \symbol{"E003}  \symbol{"E13B} \symbol{"E13C}
\end{itemize}
\end{exampleblock}
\end{frame}


\begin{frame}{Example blocks}
\begin{columns}[T]
\begin{column}{0.45\textwidth}
\begin{exampleblock}{Transparent uncover}
  \setbeamercovered{transparent}
  \begin{itemize}[<+->]
  \item Item 1
  \item Item 2
  \item Item 3
  \end{itemize}
\end{exampleblock}
\end{column}


\begin{column}{0.45\textwidth}
\begin{exampleblock}{Highlighting}
\begin{enumerate}[<alert@+>]
\setbeamercolor{alerted text}{fg=green!50!black}
\item Item 1
\item Item 2
\item Item 3
\end{enumerate}
\end{exampleblock}
\end{column}
\end{columns}
\end{frame}



\section{Code listings}

\begin{frame}[fragile]
\frametitle{Example of \R code}
Some code
\begin{Sinput}
binary <- function(i){
    a <- 2^(0:9); b <- a;
    lista <- list(NULL,NA,1,2,3,T,F)
    # Typeface for code is monospaced, comments are \LaTeX{}, slanted Roman (with math support: $\sqrt{x_{i}}$)
    function(x) sum(10^(0:9)[(x %% 2) != 0])
    print('a string')
}
\end{Sinput}

Some output of another \R{} command: \lil|print(...)|\\
\begin{Soutput}
 [1]  1  2  3  4  5  6  7  8  9 10
\end{Soutput}
% See http://texblog.org/2013/03/14/menukeys-typesetting-menu-sequences-directory-path-names-and-keyboard-shortcuts-in-latex/
For saving a file to the desktop, go to \menu[,]{File, Save As...} and navigate to \directory{Users/Username/Desktop}.\\
\medskip
Alternatively, use: \keys{\ctrl+\shift+S} and \keys{\ctrl+\return+\tab}.
\end{frame}



\section{Formulas}
\begin{frame}
\frametitle{Some impressive-looking formulas\ldots}
\begin{itemize}[<+->]
\item Meaningless expression:
\[
\frac{\partial \mathcal{L} \left(R(t), \phi\right)}{\partial t} \sim
\left \lbrace \int_{-\infty}^{+\infty} \frac{\widetilde{\mathcal{H}}(\gamma) \cdot \sqrt{\lim_{t\rightarrow\infty}\exp^{t-1}}}{\sum_{i=1}^\infty \mathbb{E}\left( \dot{x}_i^2\right)} dt \right \rbrace^{-1}
\]
\item Matrices and greek letters
\[
\forall \Gamma \Delta\exists \Theta \otimes \mathbb{R} \succeq
\begin{bmatrix}
\alpha & \beta  & \gamma & \delta  & \varepsilon & \zeta & \eta   \\
\theta & \iota  & \kappa & \lambda & \mu         & \xi   & \pi    \\
\rho   & \sigma & \tau   & \phi    & \chi        & \psi  & \omega \\
\end{bmatrix}
=
\begin{bmatrix}
a_{\ell \kappa} & \ldots & \tau_{\ell i} \\
\vdots          & \ddots & \vdots        \\
\tau_{yj}       & \ldots & a_{nn}        \\
\end{bmatrix}
\]

\[
\dot{a}\; \dot{x}\; \dot{y}\; \dot{w}\; \dot{z} \;\dot{\tau}
\]

% \[
% \dot{A}\; \dot{X}\; \dot{Y}\; \dot{W}\; \dot{Z} \;\dot{\Tau}
% \]
\end{itemize}
\end{frame}



\section{Plots}
\begin{frame}[fragile]
\frametitle{A plot}
\begin{center}
\begin{tikzpicture}
\begin{axis}[
  axis x line=middle,
  axis y line=left,
  domain=0:2,
  legend pos= north west,
  xlabel=$x$,
  ylabel=$ f(x)$,
  ytick={0,1,2,3},
  every axis y label/.style={at={(current axis.above origin)},anchor=north east}
]
\addplot[blue,mark=none,domain=0:2.01,samples=101] {(x^2)};
\addlegendentry{$x^2$}
\only<2>{
\addplot[red,mark=none,domain=0:2,samples=101] {sqrt(x)};
%\addplot[red,mark=none] gnuplot[domain=0:2,samples=101] {sqrt(x)};
\addlegendentry{$\sqrt{x}$}
}
\end{axis}
\end{tikzpicture}
\end{center}
\end{frame}


\begin{frame}[fragile]
\frametitle{A plot}
\begin{center}
\begin{tikzpicture}
\begin{axis}[
  height=5cm, width=10cm,
  xmin=-3.5, xmax=3.5,
  ymin=0, ymax=0.50,
  ytick={0},
  axis x line=bottom, axis y line=center,
  xlabel=$x$, ylabel=$f(x)$,
  every axis x label/.style={at={(current axis.right of origin)},anchor=north west},
  every axis y label/.style={at={(current axis.above origin)},anchor=north west}
]

% Draw normal curve
\addplot[blue,thick,smooth,domain=-3.5:3.5] {
  (1/(sqrt(2*pi)))*exp(-(x^2)/2) 
};

% Draw vertical line:
\only<2->{
  \draw [red,thick] (axis cs:-1.5,0) -- (axis cs:-1.5,0.1295176);
}

% Draw area
\only<3->{
  \addplot[fill=blue,opacity=0.4,smooth,domain=-3.225:-1.51] { (1/(sqrt(2*pi)))*exp(-(x^2)/2) } \closedcycle;
}
\end{axis}
\end{tikzpicture}

\end{center}
\end{frame}



\section{Tabulars}

\begin{frame}
\frametitle{Tables}
\begin{center}
\begin{tabular}{ccc}
\toprule
Title left &          & Title right \\
\midrule \pause
Left 1     & Center 1 & Right 1     \\
Left 2     & Center 2 & Right 2     \\
\bottomrule
\end{tabular}
\end{center}
\end{frame}



\section{References}

\begin{frame}
\frametitle{References}
\begin{thebibliography}{9}
\setbeamertemplate{bibliography item}[book]
\bibitem{1} Nelson, R. R., and Winter, S. G. 1982. \emph{An Evolutionary Theory of Economic Change.} Cambridge, Massachusetts: Belknap press.
\setbeamertemplate{bibliography item}[article]
\bibitem{2} Friedman, M. 1953. ``The Methodology of Positive Economics.'' In \emph{Essays in Positive Economics.} Chicago: University of Chicago Press.
\bibitem{3} Alchian, A. A. 1950. ``Uncertainty, Evolution and Economic Theory.'' \emph{Journal of Political Economy} 58: 211--222.
\setbeamertemplate{bibliography item}[book]
\bibitem{4} Koopmans, T. C. 1 957. \emph{Three Essays on the State of Economic Science.} New York: McGraw-Hill.
\end{thebibliography}
\end{frame}

\appendix

\begin{frame}[label=Hayek-quote]{Hayek's problem\hfilll \hyperlink{Hayek-back}{\beamerreturnbutton{Literature}}}
\begin{block}{}
\begin{columns}
\begin{column}{0.25\textwidth}
\includegraphics[width=\textwidth]{./Images/Hayek.jpg}
\end{column}
\begin{column}{0.7\textwidth}
\begin{myquote}
\lettrine[findent=3.5pt,nindent=-3pt]{T}{he} peculiar character of the problem of a rational economic order is determined precisely by the fact that the \highl{knowledge of the circumstances of which we must make use never exists in concentrated or integrated form}, but solely as the dispersed bits of incomplete and frequently contradictory knowledge which all the separate individuals possess.
\end{myquote}
\end{column}
\end{columns}
%\flushright Hayek (1945), \emph{The use of Knowledge in Society}\only<2>{, \highld{emphasis added}}
\end{block}
\end{frame}

\end{document}
