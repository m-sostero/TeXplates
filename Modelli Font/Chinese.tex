\documentclass[12pt]{article}
\usepackage{fontspec} 
\usepackage{xeCJK} 
%\XeTeXlinebreaklocale "zh" 
%\XeTeXlinebreakskip = 0pt plus 1pt 

\setmainfont[Mapping=tex-text]{Times New Roman} % rm
\setsansfont[Mapping=tex-text]{Arial}           % sf
\setmonofont{Courier New}                       % tt
\setCJKmainfont{DFKai-SB} %xelatex 標楷體
\setCJKmonofont{MingLiU}  %xelatex 細明體
\title{千字文}
\begin{document}

\maketitle

\begin{center}
\begin{large}
天地玄黃 宇宙洪荒\\
日月盈昃 辰宿列張\\
寒來暑往 秋收冬藏\\
閏餘成歲 律召調陽\\
雲騰致雨 露結為霜\\
金生麗水 玉出崑岡\\
劍號巨闕 珠稱夜光\\
果珍李柰 菜重芥薑\\
海鹹河淡 鱗潛羽翔\\
龍師火帝 鳥官人皇\\
始制文字 乃服衣裳\\
推位讓國 有虞陶唐\\
弔民伐罪 周發殷湯\\
坐朝問道 垂拱平章\\
愛育黎首 臣伏戎羌\\
遐邇壹體 率賓歸王\\
鳴鳳在樹 白駒食場\\
化被草木 賴及萬方\\
蓋此身髮 四大五常\\
恭惟鞠養 豈敢毀傷\\
女慕貞絜 男效才良\\
知過必改 得能莫忘
\end{large}
\end{center}
\end{document}