%\documentclass[ps,CFframes_inst]{beamer}
\documentclass[handout,a4paper]{beamer} 			%Option 2: Slides-based handouts
\usepackage{pgfpages}									%Option 2 
\pgfpagesuselayout{4 on 1}[a4paper,landscape,border shrink=2mm]	%Option 2
\usepackage{beamerprosper}
\usepackage{tikz}
\usepackage[utf8]{inputenc}
\usepackage{graphicx}
\usepackage{amsmath,mathptm}

\mode<handout>
{
  \usetheme{default}
  % or ...

  \usecolortheme{crane}
  
  \setbeamercovered{transparent}
  % or whatever (possibly just delete it)
}

\pgfdeclareimage[width=1.6cm]{university-logo}{tondo_gray}
\logo{\pgfuseimage{university-logo}}

% \documentclass[pdf,slideColor,colorBG]{prosper}
% \usepackage[mac]{inputenc}
% \usepackage{pstricks,pst-node,pst-text,pst-3d}
% \usepackage{amsmath}


\title{Computational tools for \\economics and management}
\subtitle{2009/2010}

\author{Paolo Pellizzari}

\begin{document}
\maketitle

% ---
\begin{frame}
    \frametitle{Computational tools}
    \begin{itemize}
        \item Instructor
    
        \item Course
    
        \item Contents
	
	\item Exam
	
	\item Miscellanea
    \end{itemize}
\end{frame}
% ---

% ---
\section{Instructor}
\begin{frame}
    \frametitle{CompTools: instructor}

\begin{itemize}
	    \item Paolo Pellizzari: \texttt{paolop@unive.it}
	
  	    \item Student time \begin{tabular}{ccc}
  	        
  	         & Day & Time  \\
		 \hline  	        
  	        VE (S. Giobbe C/32) & Friday & 12.30 - 14.00  \\
  	        
  	        TV (S. Paolo) & Tuesday & 12.00 - 13.30
  	    \end{tabular}
	    
	    \item \alert{There are exceptions due to relocation of the
	    DMA}.  All changes to the schedule are published.
	    
	    \item Lectures \medskip
	    
	    \begin{tabular}{c|l}
	        
	        Wednesday & 10.30-12.00, S. Giobbe 3A  \\
	        
	        Thursday & 10.30-12.00, S. Giobbe 3A \\
	        
	        Friday & 10.30-12.00, S. Giobbe 3A \\
	        
	         
	    \end{tabular}
	\end{itemize}
\end{frame}
% ---

% ---
\subsection{Research}
\begin{frame}
    \frametitle{My research}

\begin{itemize}
	    \item<1-> Monte Carlo simulation
	    	    
	    \item<2-> Derivative assets and option pricing
	    
	    \item<3-> Artificial markets \alert<3>{(you may be 
	    involved, more to come\ldots)}
	    
	    \item<4-> Agent-based models in economics and finance
	    
	\end{itemize}
\end{frame}
% ---

% ---
\section{Contents}
\begin{frame}
    \frametitle{CompTools: contents}
 \begin{itemize}
     
     \item[] Topics
     
	 \item Introduction to Computational Economics
     
         \item Ramsey growth model: Excel
	 
	 \item Neural networks: Excel
	 
	 \item Optimal control: GAMS
	 
	 \item Portfolio selection (and Genetic Algorithms): \texttt{R}
	 
	 \item Genetic Algorithms (and portfolio selection): \texttt{R}
     \end{itemize}
\end{frame}
% ---



% ---
\section{Course and exam}
\begin{frame}
    \frametitle{CompTools: course}
    \begin{itemize}
	\item Textbook: Kendrick D., Mercado, R., Amman H.,
	"Computational Economics", Princeton University Press, 2006.
	
	\item The first two chapters (included in the program) can be 
	downloaded at 
	
	\alert{Download the first chapter asap!}
	
	\item The authors' website is at

	
	\item Links to download software will be provided: GAMS and 
	\texttt{R}. Handout by the instructor.
	
	\item Online material on the web page of the course\\


	\item Handwritten notes and computer practice\ldots
     \end{itemize}
\end{frame}
% -----

% ----
\begin{frame}
    \frametitle{CompTools: examination}
    \begin{itemize}
    \item Written test with:
    \begin{enumerate}
        \item 10 multiple choices (one out of four);
    
        \item 2 open questions on models;
    
        \item 2 computer outputs to comment, solve or describe.
    \end{enumerate}
    
    \item (Probably) oral test to get 25+.
    
    \item This course will be an \emph{agony} and \emph{you'll get
    poor results} if you do not practice on a computer. Start now!

\end{itemize}
\end{frame}
% ---

% ---
\section{One more thing}
\begin{frame}
    \frametitle{CompTools: misc} 
    \begin{itemize}
	\item Universit\`a, \emph{Universitas}\ldots Studio,
	\emph{Studium}\ldots
	
	\item ``I'm interested in the future because I plan to stay
	there for the next few years''

	\item Presidenza di Facolt\`a, Faculty dean (email \texttt{presidec@unive.it}),
	Pellizzari (email \texttt{paolop@unive.it})

	\item Difensore degli studenti, students' defender:
	\texttt{difenso@unive.it}\\
% 	\texttt{\small http://venus.unive.it/urp/difensore.htm}
	
	\item Teaching evaluation forms
\end{itemize}
\end{frame}
% ---

% ---
\section{Today}
\begin{frame}
    \frametitle{Today: computational economics at large} 
    \begin{itemize}
	\item Numerical analysis applied to economic problems (say,
	linear and non-linear programming, optimization)
	
	\item Non standard computational tools for economic problems:
	simulation, data-mining, neural networks, genetic algorithms
	
	\item Analysis of ``new'' models that are almost intractable
	without computers.  Often the description of the model is
	itself an algorithm or a fragment of code.
	
	\begin{block}{Features}
	    \begin{enumerate}
	        \item Lots of agents
	    
	        \item Massive interaction
	    
	        \item Feedback effects, non-linearities and emergent 
		behaviour
	    \end{enumerate}
       \end{block}	    

    \end{itemize}
\end{frame}
% ---

% ---
\begin{frame}
    \frametitle{Today: Ramsey growth model} 
    \begin{itemize}
	\item Consumption and investment (same as consumption and 
	saving).
	
	\item Production function: capital is used to produce output.
	
	\item Capital accumulation relationship and consumption.
	
	\item Utility function

	\begin{alertblock}{Intertemporal consumption}
	    \begin{itemize}
		\item More consumption in a period means more utility\ldots
		
		\item \ldots but also less investment and, hence, 
		less consumption in the future.
	    \end{itemize}
       \end{alertblock}	    

    
    \end{itemize}
\end{frame}
% ---

% ---
\begin{frame}
    \frametitle{April 22nd: announcements} 
    \begin{itemize}
	\item Student time i(i.e. , office hours) is available in 
	Treviso. No lectures there\ldots
	
	\item Next scheduled office hours:\bigskip
	
	\begin{tabular}{lcl}
	    
	    April Tue 27th & 10.00-11.30 & Ca' Dolfin (VE)  \\
	    
	    May Tue 4th & 12.00-13.30 & Treviso  \\
	    
	    May Fri 7th & 12.30-14.00 & S. Giobbe C/32  \\
	    \hline
	    
	    \multicolumn{3}{c}{Then every Tue in TV and Fri in VE}\\
	     
	\end{tabular}
	
	\item I have uploaded the slides of the first lecture.
	
	\item Unfortunately, no other room is available.
    
    \end{itemize}
\end{frame}
% ---

% --- 28/04/2010
\begin{frame}
    \frametitle{No Ramsey\ldots no party!} 
    \begin{columns}
	\begin{column}{5cm}
	    \begin{block}{Ramsey}
		\begin{itemize}
		    \item Maximize disc. utility
		
		    \item Capital as function of capital, production 
		    and consumption
		
		    \item Terminal condition for $K$
		\end{itemize}
		\end{block}
		\end{column}
\pause		
	\begin{column}{5cm}
	    \begin{block}{Fish (natural resource)}
		\begin{itemize}
		    \item Maximize disc. cashflows
		
		    \item Fish as function of fish and fished quantity
		
		    \item Terminal condition (to avoide depletion)
		\end{itemize}
		\end{block}
		\end{column}
	\end{columns}

\end{frame}
% ---

% --- 28/04/2010
\begin{frame}
    \frametitle{Mine management} 
    \begin{columns}
	\begin{column}{5cm}
	    \begin{block}{Ramsey}
		\begin{itemize}
		    \item Maximize disc. utility
		
		    \item Capital as function of capital, production 
		    and consumption
		
		    \item Terminal condition for $K$
		\end{itemize}
		\end{block}
		\end{column}
\pause		
	\begin{column}{5cm}
	    \begin{block}{Oil extraction}
		\begin{itemize}
		    \item Maximize disc. cashflows net of costs
		
		    \item Oil left in the mine as function of oil and 
		    extracted quantity
		
		    \item Terminal condition (zero)
		    
		    \item Costs depend on the oil left:
		    $$
		    c(s,x)=-\frac{x^{2}}{s},
		    $$
		    where $s$ is left and $x$ is extracted.
		    
		\end{itemize}
		\end{block}
		\end{column}
	\end{columns}

\end{frame}
% ---



% schelling from now on

% ----
\overlays{7}{
\begin{slide}{Schelling's segregation model}
    \begin{itemstep}
	\item Why do we have segregated neighborhoods even though
	people are not racist?  (Harlem, China Town, Little Italy,
	Scampia, Brancaccio)
	
	\item ``I'm not racist!  I just prefer to live where the
	majority of residents is similar to me''
	
	\item What's the effect of this weak preference at the micro
	level?
	
	\item The model:
	\begin{enumerate}
	    \item Red and green agents live in a square grid;
	
	    \item They change location if more than 60\% of neighbors
	    are different;
	
	    \item There is no other rule.
	\end{enumerate}
	
	\end{itemstep}

\end{slide}
}
% ----- 

% ----
\overlays{5}{
\begin{slide}{Micromotives and macrobehaviour}
    \begin{itemstep}
	\item A weak personal preference is not racism but\ldots
	
	\item When you move you trigger further movements ending in strongly 
	polarized situations.
			
	\onlySlide*{3}{\item ``We should help our brothers first and the foreigner is not my
	brother, his skin has a different color\ldots Migrants?
	That's a pity but the crematorium at Santa Bona is not ready
	yet.''
	
	\medskip P. Stiffoni, MP}
	
	\onlySlide*{4}{\item ``Spreading hatred is a crime, not an
	expression of freedom of speech that must be protected.''
	
	\medskip G. Stella, Magazine Corriere della Sera, 17/06/04}

	\onlySlide*{5}{\item ``It's as if crowds are always wrong and
	individuals are always right.  However, much care is needed to
	avoid deducting any rule of conduct from this observation.''
	
	\medskip Boris Vian			}
	
	\end{itemstep}

\end{slide}
}
% ----- 




\end{document}
