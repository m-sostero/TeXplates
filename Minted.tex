%!TEX program = xelatex
%!TEX options = --shell-escape 
%!TEX options = --8bit

\documentclass[10pt,a4paper]{article}
% call (Xe)LaTeX with
% "--shell-escape" (to call Python/pygments)
% "--8bit" to display tabs correctly in XeLaTeX

\usepackage{fontspec}
\setmainfont{Times New Roman}
\setmonofont{Consolas}

\usepackage{minted}

\title{Code Highlighting with Minted}
\author{Matteo Sostero}
\begin{document}

\maketitle
Pygments 
\begin{enumerate}

\item C♯
\begin{minted}[
  mathescape,%
  linenos,%
  numbersep=5pt,%
  frame=lines,%
  framesep=2mm%
]{csharp}
string title = "This is a Unicode π in the sky"
/*
 Defined as $\pi=\lim_{n\to\infty}\frac{P_n}{d}$ where $P$ is the perimeter of an $n$-sided
 regular polygon circumscribing a circle of diameter $d$.
*/
const double pi = 3.1415926535
\end{minted}

\item C
\begin{minted}[
  linenos,
  numbersep=5pt,%
  frame=lines,%
  framesep=2mm,%
  tabsize=4,obeytabs%
]{c}
/* comment */
int main() {
hello!
	printf("hello, world");
	return 0;
}
\end{minted}

\item R
\begin{minted}[
  linenos,%
  numbersep=5pt,%
  frame=lines,%
  framesep=2mm%
]{r}
library("hello")
# Comment
x <- runif(100)
hist(x)
\end{minted}

\item Python
\begin{minted}[
  linenos,%
  numbersep=5pt,%
  frame=lines,%
  framesep=2mm,%
  tabsize=4,obeytabs%
]{python}
# Find the Fibonacci numbers
def fib(n):
    a, b = 0, 1
    while a < n:
        print(a, end=' ')
        a, b = b, a+b
    print()
fib(1000)
\end{minted}
\end{enumerate}

\end{document}
